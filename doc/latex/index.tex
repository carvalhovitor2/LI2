Pedro Miguel Ribeiro Ferreira -\/ A93282

Rui Miguel Borges Braga -\/ A93228

Vitor de Almeida de Carvalho -\/ A90648

\subsection*{L\+I2 -\/ P\+L7 -\/ G\+R\+U\+PO 6 }

\subsection*{Modus Operandi }

As alterações devem ser feitas sempre nos modulos destinados para tais funções, ou seja\+:

Funções que mexem com a lógica do programa ficam em D\+E\+F\+I\+N\+I\+D\+AS logic/logic.\+c e D\+E\+C\+L\+A\+R\+A\+D\+AS em \hyperlink{logic_8h}{logic/logic.\+h}

Funções que mexem com a interface do programa ficam em D\+E\+F\+I\+N\+I\+D\+AS interface/interface.\+c e D\+E\+C\+L\+A\+R\+A\+D\+AS em \hyperlink{interface_8h}{interface/interface.\+h}

Funções que mexem com os dados do programa ficam em D\+E\+F\+I\+N\+I\+D\+AS data/data.\+c e D\+E\+C\+L\+A\+R\+A\+D\+AS em \hyperlink{data_8h}{data/data.\+h}

\subsection*{Compilação }

Para compilar, execute o comando \textquotesingle{}make\textquotesingle{} de dentro do diretorio base do projeto. Isto é, em L\+I2\+P\+L7\+G6.

O output gerado pode ser algo como\+: 
\begin{DoxyCode}
gcc -o game main\_game.c data/data.c interface/interface.c logic/logic.c
\end{DoxyCode}
 